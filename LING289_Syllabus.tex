\documentclass[12pt]{article}

\usepackage{fontspec}
\usepackage{geometry}
\usepackage{lastpage}
\usepackage{fancyhdr}
\usepackage{hyperref}
\usepackage{enumitem}

\geometry{top=0.7in, bottom=1in, left=0.7in, right=0.7in, marginparsep=4pt, marginparwidth=1in}

\renewcommand{\headrulewidth}{0pt}
\pagestyle{fancyplain}
\fancyhf{}
\cfoot{\thepage\ of \pageref{LastPage}}

\setlength{\parindent}{0pt}
\setlength{\parskip}{0pt}

% \setromanfont [Ligatures={Common}, Numbers={OldStyle}, Variant=01,
%  BoldFont={LinLibertine_RB.otf},
%  ItalicFont={LinLibertine_RI.otf},
%  BoldItalicFont={LinLibertine_RBI.otf}
%  ]{LinLibertine_R.otf}

\usepackage{tikz}
\def\checkmark{\tikz\fill[scale=0.4](0,.35) -- (.25,0) -- (1,.7) -- (.25,.15) -- cycle;}

\usepackage{xunicode}
\defaultfontfeatures{Mapping=tex-text}

\setromanfont{YaleNew}

\begin{document}

\begin{center}
{\bf MATH/LING 289: Introduction to Data Science, Fall 2018} \\
Tuesday, Thursday 13:30-14:45 \quad PURH G13\\
\end{center}

\bigskip

\noindent
\begin{tabular}{ l l }
{\bf Instructor:} &  {\bf Taylor Arnold} \\
E-mail: & \href{mailto:tarnold2@richmond.edu}{tarnold2@richmond.edu} \\
Office: & Jepson Hall, Rm 218 \\
Office hours: & to be determined \\
Course Website: & \url{https://statsmaths.github.io/stat289-f18}
\end{tabular}

\vspace{0.5cm}

\textbf{Overview:} \vspace{6pt}

Data science is an interdisciplinary field concerned with extracting knowledge
from data and communicating those results to some public audience. Data
science needs to be learned \textit{by doing} data science. There are
no short-cuts and the process cannot be learned by simply working our way
methodically through a textbook of disconnected topics.\\

Therefore, this course will be taught using a problem-based learning model. As
a class, we will be addressing an open ended research question and learning
various skills that will assist our inquiry. Both individually and in small
groups, students will take responsibility for particular subtasks that drive
our research forward. At the end of the semester students will have acquired
a toolkit of methods, and the knowledge of how to use them in practice, to
address important social, cultural, and scientific questions with data-driven
techniques.\\

We will make heavy use of computing throughout the semester, but \textit{no
prior programming experience is required}. Also note that this course has a
MATH designation because statistics is currently housed in the mathematics
department. The topics of this course do \textit{not} fall within the
traditional disciplinary boundaries of mathematics.\\

\textbf{Research Question:} \vspace{6pt}

Our object of study for this semester (Fall 2018) concerns the edit history
of pages on Wikipedia. Our research for the semester will be guided by several
related questions:

\begin{itemize}\setlength\itemsep{0em}
\item How is knowledge being represented through the works of an anonymous,
decentralized collective of users connected across the internet?
\item What role does memory play in the representation of current and
semi-current events?
\item What kinds of knowledge are privileged, or taken for granted, by
citation patterns on the interent?
\item How do cultural and linguistic factors play into the structure of
Wikipedia pages (by looking at pages across different languages)?
\item What role do images and other media play in the structure and
development of encyclopedic pages? has this changed over time?
\end{itemize}

These questions can be studied from a number of disciplinary perspectives.
For example, one might draw on methods from from one or more of: psychology,
sociology, linguistics, political science, American studies, media studies,
and critical theory. In this course we will see how the methods of data
science provide a new set of tools that are able to engage with, rather than
against, these disciplinary techniques while opening the possibility of
producing knowledge through the study of large unstructured data sets.

\clearpage

\textbf{Grades:} \vspace{6pt}

Your final grade will consist of three elements, weighted as follows:

\begin{itemize}[noitemsep,topsep=0pt]
\item Class Participation, 20\%
\item Short Paper (1000-1500 words), 20\%
\item Final Paper (2000-2500 words), 60\%
\end{itemize}

Note that both papers will also include graphics and models constructed from
our data. Course expectations and community standards will be discussed,
developed and distributed in the first week of the course. This will include
policies for class participation, attendance, and late work.

\vspace{0.5cm}

\textbf{Weekly Topics:} \vspace{6pt}

Typically, we will spend Tuesdays discussing the course readings and
introducing new concepts. Thursday will be spent working on coding exercises
and small group discussions.

\vspace{0.5cm}

%%%%%%%%%%%%%%%%%%%%%%%%%%%%%%%%%%%%%%%%%%%%%%%%%%%%%%%%%%%%%%%%%%%%%%%%%%%%%%
\underline{Week 1: Introduction, What is Data Science, Set-up}

\begin{itemize}[noitemsep,topsep=0pt,leftmargin=!,labelindent=5pt,itemindent=-70pt]
\item Reading \#1: ``Data Analysis and Statistics: An Expository Overview."
J. W. Tukey and M. B. Wilk. \textit{AFIPS Conference Proceedings}. Volume 29,
1966.
\end{itemize}

\vspace{12pt}

%%%%%%%%%%%%%%%%%%%%%%%%%%%%%%%%%%%%%%%%%%%%%%%%%%%%%%%%%%%%%%%%%%%%%%%%%%%%%%
\underline{Week 2: Keywords in Context and Basic Search}

\begin{itemize}[noitemsep,topsep=0pt,leftmargin=!,labelindent=5pt,itemindent=-70pt]
\item Reading \#1:
\end{itemize}

\vspace{12pt}

%%%%%%%%%%%%%%%%%%%%%%%%%%%%%%%%%%%%%%%%%%%%%%%%%%%%%%%%%%%%%%%%%%%%%%%%%%%%%%
\underline{Week 3: Corpus Construction}

\begin{itemize}[noitemsep,topsep=0pt,leftmargin=!,labelindent=5pt,itemindent=-70pt]
\item Reading \#1:
\end{itemize}

\vspace{12pt}

%%%%%%%%%%%%%%%%%%%%%%%%%%%%%%%%%%%%%%%%%%%%%%%%%%%%%%%%%%%%%%%%%%%%%%%%%%%%%%
\underline{Week 4: Term-Frequency Matrices}

\begin{itemize}[noitemsep,topsep=0pt,leftmargin=!,labelindent=5pt,itemindent=-70pt]
\item Reading \#1:
\end{itemize}

\vspace{12pt}

%%%%%%%%%%%%%%%%%%%%%%%%%%%%%%%%%%%%%%%%%%%%%%%%%%%%%%%%%%%%%%%%%%%%%%%%%%%%%%
\underline{Week 5: Dimensionality Reduction}

\begin{itemize}[noitemsep,topsep=0pt,leftmargin=!,labelindent=5pt,itemindent=-70pt]
\item Reading \#1:
\end{itemize}

\vspace{12pt}

%%%%%%%%%%%%%%%%%%%%%%%%%%%%%%%%%%%%%%%%%%%%%%%%%%%%%%%%%%%%%%%%%%%%%%%%%%%%%%
\underline{Week 6: Network Analysis}

\begin{itemize}[noitemsep,topsep=0pt,leftmargin=!,labelindent=5pt,itemindent=-70pt]
\item Reading \#1:
\end{itemize}

\vspace{12pt}

%%%%%%%%%%%%%%%%%%%%%%%%%%%%%%%%%%%%%%%%%%%%%%%%%%%%%%%%%%%%%%%%%%%%%%%%%%%%%%
\underline{Week 7: Token Annotations}

\begin{itemize}[noitemsep,topsep=0pt,leftmargin=!,labelindent=5pt,itemindent=-70pt]
\item Reading \#1:
\end{itemize}

\vspace{12pt}

%%%%%%%%%%%%%%%%%%%%%%%%%%%%%%%%%%%%%%%%%%%%%%%%%%%%%%%%%%%%%%%%%%%%%%%%%%%%%%
\underline{Week 8: Phrase-level Linguistic Annotations}

\begin{itemize}[noitemsep,topsep=0pt,leftmargin=!,labelindent=5pt,itemindent=-70pt]
\item Reading \#1:
\end{itemize}

\vspace{12pt}

%%%%%%%%%%%%%%%%%%%%%%%%%%%%%%%%%%%%%%%%%%%%%%%%%%%%%%%%%%%%%%%%%%%%%%%%%%%%%%
\underline{Week 9: Topic Models and Document Clustering}

\begin{itemize}[noitemsep,topsep=0pt,leftmargin=!,labelindent=5pt,itemindent=-70pt]
\item Reading \#1:
\end{itemize}

\vspace{12pt}

%%%%%%%%%%%%%%%%%%%%%%%%%%%%%%%%%%%%%%%%%%%%%%%%%%%%%%%%%%%%%%%%%%%%%%%%%%%%%%
\underline{Week 10: Image Processing}

\begin{itemize}[noitemsep,topsep=0pt,leftmargin=!,labelindent=5pt,itemindent=-70pt]
\item Reading \#1:
\end{itemize}

\vspace{12pt}



\end{document}





