\documentclass[12pt]{article}

\usepackage{fontspec}
\usepackage{geometry}
\usepackage{lastpage}
\usepackage{fancyhdr}
\usepackage{hyperref}
\usepackage{enumitem}

\geometry{top=0.7in, bottom=1in, left=0.7in, right=0.7in, marginparsep=4pt, marginparwidth=1in}

\renewcommand{\headrulewidth}{0pt}
\pagestyle{fancyplain}
\fancyhf{}
\cfoot{\thepage\ of \pageref{LastPage}}

\setlength{\parindent}{0pt}
\setlength{\parskip}{0pt}

% \setromanfont [Ligatures={Common}, Numbers={OldStyle}, Variant=01,
%  BoldFont={LinLibertine_RB.otf},
%  ItalicFont={LinLibertine_RI.otf},
%  BoldItalicFont={LinLibertine_RBI.otf}
%  ]{LinLibertine_R.otf}

\usepackage{tikz}
\def\checkmark{\tikz\fill[scale=0.4](0,.35) -- (.25,0) -- (1,.7) -- (.25,.15) -- cycle;}

\usepackage{xunicode}
\defaultfontfeatures{Mapping=tex-text}

\setromanfont{YaleNew}

\begin{document}

\begin{center}
{\bf MATH/LING 289: Introduction to Data Science, Fall 2018} \\
Tuesday, Thursday 13:30-14:45 \quad PURH G13\\
\end{center}

\bigskip

\noindent
\begin{tabular}{ l l }
{\bf Instructor:} &  {\bf Taylor Arnold} \\
E-mail: & \href{mailto:tarnold2@richmond.edu}{tarnold2@richmond.edu} \\
Office: & Jepson Hall, Rm 218 \\
Office hours: & to be determined \\
Course Website: & \url{https://statsmaths.github.io/stat289-f18}
\end{tabular}

\vspace{0.5cm}

\textbf{Overview:} \vspace{6pt}

Data science is an interdisciplinary field concerned with extracting knowledge
from data and communicating those results to some public audience. Data
science needs to be learned \textit{by doing} data science. There are
no short-cuts and the process cannot be learned by simply working our way
methodically through a textbook of disconnected topics.\\

Therefore, this course will be taught using a problem-based learning model. As
a class, we will be addressing an open ended research question and learning
various skills that will assist our inquiry. Both individually and in small
groups, students will take responsibility for particular subtasks that drive
our research forward. At the end of the semester students will have acquired
a toolkit of methods, and the knowledge of how to use them in practice, to
address important social, cultural, and scientific questions with data-driven
techniques.\\

We will make heavy use of computing throughout the semester, but \textit{no
prior programming experience is required}. Also note that this course has a
MATH designation because statistics is currently housed in the mathematics
department. The topics of this course, however, do not fall within the
traditional disciplinary boundaries of mathematics.\\

\textbf{Research Question:} \vspace{6pt}

Our object of study for this semester (Fall 2018) concerns the edit history
of pages on Wikipedia. Our research for the semester will be guided by drawing
on questions such as:

\begin{itemize}\setlength\itemsep{0em}
\item How is knowledge being represented through the works of an anonymous,
decentralized collective of users connected across the internet?
\item What role does memory play in the representation of current and
semi-current events?
\item What kinds of knowledge are privileged, or taken for granted, by
citation patterns on the interent?
\item How do cultural and linguistic factors play into the structure of
Wikipedia pages (by looking at pages across different languages)?
\item What role do images and other media play in the structure and
development of encyclopedic pages? has this changed over time?
\end{itemize}

These questions can be studied from a number of disciplinary perspectives.
For example, one might draw on methods from from one or more of: psychology,
sociology, linguistics, political science, American studies, media studies,
and critical theory. In this course we will see how the methods of data
science provide a new set of tools that are able to engage with, rather than
against, these disciplinary techniques while opening the possibility of
producing knowledge through the study of large unstructured data sets.

\clearpage

\textbf{Grades:} \vspace{6pt}

Your final grade will consist of three elements, weighted as follows:

\begin{itemize}[noitemsep,topsep=6pt]
\item Class Participation, 20\%
\item Project Prospectus, 20\%
\item Final Project and Presentation, 60\%
\end{itemize}

\vspace{3pt}

Course expectations and community standards will be discussed, developed and
distributed in the first week of the course. This will include policies for
class participation, attendance, and late work.

\vspace{0.5cm}

\textbf{Texts:} \vspace{6pt}

Readings for the course, which will help us address and position our research,
will be pulled from open access journal articles and the following required
text:

\begin{quote}
Jemielniak, Dariusz, 2014. \textit{Common Knowledge?: An Ethnography
of Wikipedia.} Stanford University Press.
\end{quote}

As a reference for technical topics, I will make frequent reference to:

\begin{quote}
Wickham, Hadley and Grolemund, Garrett, 2016.
\textit{R for Data Science: Import, Tidy, Transform, Visualize, and Model
Data.} O'Reilly Media.
\end{quote}

This second text as it is available online
(\url{http://r4ds.had.co.nz/}) for free and in its entirety.

\vspace{0.5cm}

\textbf{Computing:} \vspace{6pt}

We will use the \textbf{R} programming environment throughout the
semester. It is freely available for all major operating systems and
is pre-installed on many campus computers. No prior programming experience is
assumed or required for this course. We will install the required software
on your computer during the first week of the course (alternatively, R is
installed in several computer labs across campus).

\vspace{0.5cm}

\textbf{Data Science Topics:} \vspace{6pt}

Given the project-oriented aspect of this course, exact schedule of topics
will depend on and adapt to the natural course of our investigation. Technical
data-science topics that may be covered during the semester include:

\begin{itemize}[noitemsep,topsep=6pt]
\item data structures for storing data
\item visualization techniques for exploratory analysis
\item web scrapping
\item document summarization
\item sentence parsing
\item named entity recognition (NER)
\item automated geocoding
\item dimensionality reduction
\item topic models
\item network analysis
\item image processing
\end{itemize}

\vspace{3pt}

Strong consideration will be given to the specific of interests and
motivations of students enrolled in the course.

\end{document}





